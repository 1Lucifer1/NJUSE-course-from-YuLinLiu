\documentclass[a4paper,UTF8]{article}
\usepackage{ctex}
\usepackage[margin=0.75in]{geometry}
\usepackage{color}
\usepackage{graphicx}
\usepackage{amssymb}
\usepackage{amsmath}
\usepackage{amsthm}
\usepackage{tcolorbox}
\usepackage{enumerate}
\usepackage{hyperref}
%\usepackage[thmmarks, amsmath, thref]{ntheorem}
\theoremstyle{definition}
\newtheorem*{solution}{Solution}
\usepackage{multirow}              
\usepackage{ulem}
\newcommand{\li}{\uline{\hspace{0.5em}}}
\newcommand\rsx[1]{\left.{#1}\vphantom{\Big|}\right|}
\usepackage{subfigure}
\usepackage{listings}
\usepackage{hyperref}
\hypersetup{hypertex=true,
	colorlinks=true,
	linkcolor=blue,
	anchorcolor=blue,
	citecolor=blue}

\begin{document}
	\title{裂缝识别模型的测试数据度量以及扩增综述}
	\author{181250090 刘育麟}
	\date{}
	\maketitle
	\section*{摘要}
	道路上的裂缝大幅降低了车辆行驶在路面上时的行车安全性与舒适性,而在隧道里面,隧道裂缝是隧道坍塌的隐患。为了解决这个问题,现有许多论文提出了如何建立一个有效的模型来检测路面或是隧道中的裂缝。然而,目前基于深度学习的裂缝识别总是面临数据集过小的问题。因此,本片文章调研了大量的文献,分析裂缝识别的领域特殊性与难点,整理了现今使用的数据度量指标与数据扩增方法,并在最后提出裂缝检测模型的未来期望,为此方面的研究者提供参考。
	
	\textbf{关键词:} 数据度量;裂缝识别;数据扩增;数据增强;卷积神经网络
	
	\section{引言}
	路面上的裂缝是在交通上面很大的一个安全隐患。道路裂缝容易造成车辆行驶的问题、而隧道裂缝容易引起坍塌与结构破坏,所以修补裂缝与维持路面表面状态良好是交通管理部门一项重要工作。过往的检测方法是使用裂缝观测仪,人工进行判读与记录,此方法费时而且主观性强,容易造成误判而导致未能及时修补裂缝,造成不可挽回的后果\cite{2}。
	
	近年来,随着科技的不断进步,机器学习成为了当今最火热的课题,而道路的裂缝检测也因为机器学习的发展而不断地建立基于深度学习的模型来检测路面裂缝,基于神经网络的路面裂缝检测已经是主流趋势。Hyunwoo Cho提出使用CWT算法来进行裂缝识别\cite{3};Zhang 等人提出了ADA3D裂缝识别算法\cite{4};沙爱民等利用3个CNN模型分别完成路面病害识别、路面裂缝特征提取和坑槽特征提取\cite{5},上述的几个都是透过深度学习中的卷积神经网络技术(Convolutional Neural Network, CNN)搭建深度学习模型,该算法能够有效的避免过去传统机器学习的复杂且耗时的人工提取特征,它能够自动的进行特征学习、辨识特征,因此广泛的用于有关于图像分类、目标识别的机器学习任务中。
	
	深度学习算法模型需要大量的参数进行训练才能得到较好结果,而当前道路工作者所能采集到的数据有限,而且受限于当前环境,像是光照、角度、表面颗粒不均而会产生大量的噪声,数据必须经过处理才能使用。稀少的数据会造成模型的欠拟合,因此,调研近年来的关于裂缝识别的论文,并且对学术论文中有关数据处理与数据扩增的方式进行探讨,归纳之后分为几个部分进行阐述:
	\begin{enumerate}[(1)]
		\item 领域特殊性与数据特殊要求
		\item 数据度量方式与扩增和增强方法
		\item 未来可能的研究方向
	\end{enumerate}
	
	\section{特征分析}
	特定领域进行机器学习时,数据特征与模型特性时必须考虑的因素。不同的领域需要的数据特征不同,特定领域中在数据较为重点关注的地方进行任何小幅度的修改,都可能影响模型最终的训练结果。所以在进行数据度量前,必须先了解当今用于裂缝识别的主流模型与数据特色,在度量时偏重这几点进行考察与处理,才能得到更适合模型的数据集。
	\subsection{主流模型}
	现今的模型大部分是基于卷积神经网络的,但是卷积神经网络的一个特点就是有监督学习。这仰赖人工标注数据特征,但是这种标注成本高而且时间久,常常需要好几个月来创建一个训练数据集。目前调研的几篇论文中,除了在2020年8月发表的基于卷积自编码图像扩增的识别\cite{1}对比了无监督学习与有监督学习的差异之外,大部分的论文仍然是基于有监督学习的方法进行训练。但是,最新的论文已经开始向这方面发展可以表明,该领域的学者已经思考如何将无监督学习转换为有监督学习来降低人工标注的成本。
	\subsection{数据特色}
	在裂缝识别技术中,训练的模型都是基于图像检测的。因此,图像扩增的经典算法诸如几何变换在这里也是适用的。但是,必须要注意的是,到目前为止,裂缝识别的图像样本并没有公开、适用于研究的裂缝数据集,这使得数据大部分来自人工采集与数据合成,而且数据大部分来源于真实的户外场景,这常常导致几个问题:
	\begin{enumerate}[(1)]
		\item 图像大小与像素差异
		\item 路面颗粒、阴影等与裂缝灰度相近的大量随机噪声
		\item 路面上光照不同导致裂缝对比度不明显
		\item 环境复杂多样性导致的噪声
	\end{enumerate}
    这些问题是讨论裂缝识别技术的论文不可避免必须要解决的难题。
    
    \subsection{裂缝定义}
    关于裂缝本身在图像数据中的存在与性质,\cite{12} 里面根据数学形态学提出的图像中裂缝的定义:
    \begin{enumerate}[定义1]
    	\item 裂缝比背景黑。裂缝在图像中的像素是局部极小值。
    	\item 裂缝是一串连续的鞍点。裂纹是一个薄的沟槽,使得裂纹的横截面呈马鞍形。
    	\item 裂缝有线性特征或方向性。一条裂缝是图像的一个连续子集,并存在一个方向。
    \end{enumerate}

    前面提出的对图像的定义过于模糊,许多论文都提出了部分较为精准与具有实际数据特征考量的定义。徐志刚等人提出来的裂缝识别算法论文\cite{6}中根据数据本身的一些噪声考量与环境因素等,点出一般裂缝图像中裂缝的性质,这几个性质对前面的定义进行了补充:
    \begin{enumerate}[性质1]
    	\item 裂缝相对于路面背景来说是一些灰度值较低的像素集合。也就是说裂缝相对于路面背景,其像素灰度值要低得多,因此裂缝上的像素值一般来说是局部最小值。
    	\item 裂缝中央灰度值较低,边缘灰度值较高,其剖面呈山谷状,是一种屋脊边缘。
    	\item 裂缝一般来说是具有一定的线性特性和一定方向,在空间上具有连续性。
    	\item 由于光照不均匀的影响,裂缝像素在整幅路面图像中的不同位置上可能呈现出不同的灰度均值。
    	\item 路面背景和裂缝目标像素灰度直方图均近似服从高斯分布。
    	\item 裂缝目标像素在整个路面图像中占据的比例很小,一般不超过20%。
    \end{enumerate}
    
    这两篇论文归纳出来的几个裂缝性质与定义可以涵盖现有技术在数据度量时所想要解决的几个问题。
    \section{数据度量指标}
    数据度量相当于对数据进行预处理,使得数据可以满足训练要求、降低噪声、凸显裂缝特征。根据调研的论文在数据处理方面的工作进行归纳,可以大概知道数据进行的预处理有大概几种:
    \subsection{灰度图片}
    图片必须为灰度图片,因为裂缝是灰度值较高的区域,所以直接将彩色去掉可以避免噪声,而且灰度图片可以减小内存负担。\cite{7}给出了数据从彩色图片变成灰色图片的公式\ref{gray_image}
    \begin{equation}
    Gray = 0.3B + 0.59G + 0.11R
    \label{gray_image}
    \end{equation}
    \subsection{图片大小}
    图片的像素必须两边相同,例如:400*400,不同长宽的图片易造成训练难度。而因为计算机内存限制,所以图片像素不建议过大,必须进行裁剪。大多数论文的像素都不超过300*300,多数是256*256。\cite{1}对数据是进行先裁剪,后缩放的处理。通过双线性插值进行横向缩放至4,000*2,000像素,对缩放后的图片进行连续剪裁并缩放,得到200*200像素的图片。
    \subsection{灰度值}
    灰度值是训练时一个很重要的特征,所以,灰度值的归一化以及降噪处理在裂缝识别中是很重要的一环。在论文的处理中,常会进行灰度增强来进行图片处理。\cite{7}使用了分段线性灰度变换来进行灰度增强。\cite{8}中对图像灰度进行Histogram equalization\cite{9},使得灰度转换对比更加明显。
    \subsection{平滑化与锐利化}
    图片中的高频区域重视细节,但是大部分的图片数据都是低频的,所以为了让图片分析器可以更好的去除图片的噪点与避免高频的干扰,图片要进行平滑化。Median filtering method\cite{10} 是\cite{8}所使用的方法。它可以很好的保护图片并且去除一些噪声。他去除了离群值的噪声并且保持数据原样,避免了特征被去掉或模糊化。
    
    为了加强图像的质地信息,我们使用图像锐利化来消除或是弱化低频部分的图像。我们需要加强图片目标区域的信息,所以外部的灰度值会趋近于零。Laplace operator是最常用于锐化图片的方法,他使用的是二次偏导数。此操作数的定义如下:
    \begin{equation}
    \nabla^2f = \partial^2 f/\partial x^2 + \partial^2 f/\partial y^2
    \end{equation}
    \subsection{阈值分割}
    部分论文\cite{3}、\cite{7}、\cite{11} 里面将数据进行黑白化,他们将根据灰度值进行分割,设置一个阈值,只要在分割的小区快里面超过这个阈值,就会将这块区域变成黑色的区域,也就是说,图片将变成白色图片中的黑色小块或黑色线条。但是考虑每个分割块也是一个问题需要考虑,他算是一个超参数,需要不断地实验来测试出最适合的大小,\cite{7}将图片切成128*128个小块。一方面,当每个分割块过小时,这样子的阈值分割会增加假阳性的侦测数量。另一方面,当分割块过大时,计算特征时小裂缝的影响容易消失。将图片进行黑白二分的方法可以将数据离散化,方便特征提取与计算。
    
    \section{数据扩增与数据增强}
    现今大部分关于裂缝检测的论文都存在一个问题,那就是数据严重不足,需要进行扩增。而因为没有统一的数据集,数据质量参差不齐,需要进行数据增强来加强模型训练精度。深度学习中的数据扩增算法可分为传统的基于图像处理的数据扩增以及基于深度学习的数据扩增\cite{13},基于深度学习的数据扩增又可以分为下面几类:
    \begin{enumerate}[(1)]
    	\item 生成式对抗网络(Generative adversary network, GAN)
    	\item 卷积自编码器。(convolutional Auto-Encoders, CAE)
    \end{enumerate}
    这几种数据扩增的方法都有论文使用,最常见的仍然是传统的数据扩增方法,基于深度学习的数据扩增仍在探索阶段,因为基于深度学习的数据扩增训练过程复杂而且计算成本庞大,而且将不同方法的鲁棒性是否足够是一大问题,但仍是现今主要的研究方向。
    \subsection{传统扩增方法}
    传统的基于图像处理的方法主要分为三大类:
    \begin{enumerate}[(1)]
    	\item 图像几何变换。简单的对图片进行裁剪、镜像、旋转、平移等变换来改变图像中目标的位置与目标的角度,产生新的测试数据。在前面的裂缝定义中提到,裂缝是具有方向性的,而且是图片中像素值的局部极小值,改变裂缝的方向与坐标能有效的扩增数据集。
    	\item 图片彩色空间变化。需要注意的是,裂缝图片数据在预处理的时候会转换成灰度图像,所以饱和度的改变在扩增裂缝数据中是没有意义的。最容易扩增数据的是改变整体图片灰度值或是部分区域灰度值,但是修改灰度值时必须考虑裂缝本身的灰度以及在进行图像预处理的阈值分割时图像对灰度的阈值,避免产生有问题的数据。而另一种方法就是改变像素点,前面的定义说明了裂缝是像素点的局部极小值,改变像素点增加噪声,可以扩增数据的同时,增加模型的鲁棒性。
    	\item 多样本合成算法。有时候,某个标签的数据会过少导致样本训练不均衡,这时候可以通过合成来增加种类少的样本。但是方法过于简单容易造成过拟合的问题,这是在扩增数据时需要考虑的问题。SMOTE算法\cite{17}就是一种多样本合成算法。
    \end{enumerate}
    传统扩增方法较为简易,可以避免花费大量的人力物力。但是机器学习的训练数据与测试数据若是没有经过良好校验或是降噪处理,使得扩增的数据在特征部分的数据相差不大,训练模型容易产生对训练数据的过拟合,降低鲁棒性,所以有许多论文尝试使用基于深度学习的方法来进行数据扩增。
    \subsection{生成式对抗网络}
    因为数据特性为图像,目前图像扩增的方法有很大一部分采用的是生成式对抗网络。生成式对抗网络主要是由生成器与判别器两部分组成。两者都是神经网络,判别器为正向、生成器为正向,判别器输入一组由生成器产生的数据,并输出这个数据为给定标签的概率值。当判别器输入真实数据时,概论应该趋近于1,而生成器的目的就是生成概论趋近于1的图片,但是判别器在判别时也会训练从而分辨生成器生成的数据是否为真。两者互相促进就是生成式对抗网络的原理。下面主要介绍两个基于GAN的数据扩增算法。
    
    Zhang K等提出了一种名为crackGAN(novel deep generative adversarial network)的算法\cite{14}来解决裂缝类型图片固有的样本不平衡问题,它引入一种基于crack-patch-only-supervision生成性对抗性损失函数,对目标函数进行正则化来进行数据质量度量,避免产生在卷积神经网络中,裂缝图片因为裂缝过于细长的关系,在某一层被识别为全部像素都是背景,导致训练出现问题,损失函数定义如下(I是图片集):
    \begin{equation}
    L_{adv} = -E_{x \in I} [LogD(G(x))]
    \end{equation}
    
    他的算法主要针对基于全卷积网络(Fully convolutional network, FCN)的裂缝识别模型以及地面的裂缝,在FCN模型中。他的算法可以让模型在训练的时候训练部分分割过的图像块,但是可以识别完整的图像。
    
    LI Liang-fu等提出了基于GAN的细小裂缝分割算法\cite{15}\cite{16}。建立基于语义分割网络的细小裂缝生成式对抗网络SE-GAN,并针对细小裂缝图像特征分别对生成式对抗网络的判别器结构和生成器结构进行调整,并引入像素损失函数。他将目标图片进行分割与超分辨率重建,并进行语义分割,得到每个小图片的结果图,最后重新拼接,得到最后检测结果。其损失函数定义如下:
    \begin{equation}
    L_{adv} = E_{x \sim P_{data(x)}}\{lb\{D\{dis[x,G(z)]\}\}\} + E_{x \sim P_{G(z)}}\{lb\{1 - D\{dis[G(z),x]\}\}\}
    \end{equation}
    
    使用了SE-GAN进行数据扩增与数据增强之后,明显可以从数据中看到模型训练效果的提升,缓解了模型因为数据质量不足的欠拟合问题,并且图像细节可以更好的被补充完整。
    \subsection{卷积自编码器}
    数据降噪和降维被认为是自编码器的两个主要实际应用。使用适当的维度和稀疏性约束,自编码器可以得到比PCA或其他类似技术更好的数据投影。而卷积自编码是传统卷积神经网络的分支,他使用卷积层与池化层替代原来的全连接层。因为全连接层在处理二维图像数据时会损失空间信息,卷积自编码器放弃堆叠数据,使图像数据输入时保持其空间信息不变,能很好的保留二维信息。
    
    现今有很多卷积自编码器在图像上的应用,Lore等\cite{18}在2017年提出低光网络LLNet,使用自编码器(auto-encoder, AE)对图像进行数据加强,该网络通过学习低光网络的特征,并利用自编码器网络的去噪能力,进行数据增强,将图像重构之后得到明亮图像,该算法在灰度图像上面具有不错的成效。之后又有许多学者基于LLNet提出新的学术成果,例如WANG Wan-liang等提出的新的基于卷积自编码器的图像增强框架CAENet\cite{19}。
    
    Yue Hou等根据卷积自编码器进行路面裂缝的数据扩增\cite{1},并研究卷积神经网络与传统扩增方法对模型的训练差异,得出CAE扩增所得到的裂缝图片更具图片特征,能够一定程度上提升数据可靠性和识别精度。而他更进一步的将传统扩增方法与CAE相结合,发现能很好的提升深度学习模型对图片的识别准确率。最后,使用的基于CAE图像特征预训练的DCEC深度聚类网络无监督学习方法,的精度比传统的有监督学习的效果更佳,拥有潜在应用场景。
    
    \section{未来研究}
    本文分析了裂缝图片数据的特征以及常用的模型,并且总结多篇论文中在数据预处理方面使用的度量指标,最后阐述当今几种对卷积神经网络模型数据的扩增方法与其在裂缝识别模型中的应用。总结发现若干未来可能的发展。
    \begin{enumerate}[(1)]
    	\item 因为裂缝图像识别裂缝时主要的判定为灰度值,因此数据在使用基于深度学习的模型进行数据增强与扩增时,可以在基于模糊测试的框架下,对灰度值进行特殊处理。这样子的处理可以避免模型过拟合,而对数据进行不同处理可以增加数据的鲁棒性与泛用性。
    	\item 将图像几何变换的传统图像处理方法与深度学习模型相结合,形成基于深度学习的图像几何、色彩变换。如此可以避免图像几何变换中一些不稳定因素导致的数据据质量下降,影响模型的准确度。而裂缝很多的特征都是与传统的图像处理方法挂钩,因此可以使用模型来训练旋转时调整的角度或是改变的灰度值,由此达到更好的数据增强与数据扩增效果。
    	\item 若是能精准的辨识裂缝位置的话,已辨识的数据可以进行重组并产生新的数据。假设一组数据被辨识为无裂缝、一组数据被归类为有裂缝,则在精准识别裂缝的框架下,可以将有裂缝标签的图片上面的裂缝与无裂缝标签上面的裂缝进行重组,可以产生大量的新数据一共训练。
    	\item 现今大部分的模型仍然基于有监督学习,耗费大量的人力。因为提出的两种基于深度学习的数据扩增方法都是无监督学习,所以可以通过数据扩增与模型训练相结合,将无监督学习数据扩增方法中识别出来的特征作为训练模型的特征传入,这样就可以建立起一个自动进行数据扩增与训练模型的图像识别模型。
    	\item 统一且泛用性高的数据集出现。本文是调研了大量的论文与研究成果中关于数据质量度量的部分总结而成,调研的所有论文都没有直接讨论如何扩增裂缝数据以及裂缝数据的质量度量,更没有统一的获取数据方式,希望未来有论文能针对这一方面进行探讨,建立起统一的度量方式与数据获取方法,为后面的相关领域研究者铺路。
    	\item 裂缝算法的通用性差。本文调研的论文所使用的裂缝数据可以分为桥梁、隧道、路面、沥青路面等大量不同类型路面的裂缝。前面总结的扩增方法缺少泛用性,大多是只适用于特定场景或是特定材质下的图像数据,缺乏鲁棒性的模型很难应用到现实生活当中,是一个亟待解决的问题。
    	\item 使用高精度图片进行训练。现今大部分的图像训练都是基于低像素的图像进行训练,像素过高容易造成数据的高维灾难,计算机的内存无法负荷训练如此高精度的图片,还需要考虑GPU 的问题,因此,图像识别技术在硬件上仍然存在的短板,设备无法支持现有理论的算法是硬伤,未来若是硬件或是算法得到了突破,则计算的精准度也会大幅度的提升。
    \end{enumerate}
    \section{总结}
    目前基于图像的自动识别裂缝的技术仍处于摸索阶段,在最近10年来不断有新的算法被应用到这个领域并取得初步成效。现今大部分论文的数据都是自行采集并扩增,互联网中并没有统一且开源的数据集供学者们使用。而现今大部分裂缝图像的采集仍需要耗费大量的人力物力去实地采集,这导致了数据的质量参差不齐以及数据的稀缺性。本文归纳了近年来关于自动裂缝识别的论文,总结出他们对数据进行质量度量与数据预处理的方法,并整理了现今几种对裂缝图像的数据增强方法,提出了该领域未来的可能发展趋势,并期待论文中的各项技术能确实落地,在工业领域中进行应用。
	\begin{thebibliography}{99}
		\bibitem{1} 侯越,陈逸涵,顾兴宇,茅荃,曹丹丹,WANG Linbing, 荆鹏. 基于卷积自编码的沥青路面目标与裂缝智能识别[J/OL].中国公路学报:1-21[2020-11-07]. http://kns.cnki.net/kcms/detail/61.1313.U.20200818.1354.002.html.
		\bibitem{2} 王睿,漆泰岳,朱鑫,李涛.隧道检测裂缝的图像处理研究[J].铁道标准设计,2014,58(10):93-96+127.
		\bibitem{3} H. Cho, H. Yoon and J. Jung, "Image-Based Crack Detection Using Crack Width Transform (CWT) Algorithm," in IEEE Access, vol. 6, pp. 60100-60114, 2018, doi: 10.1109/ACCESS.2018.2875889.
		\bibitem{4} ZHANG A, WANG K C P. Automated pixel⁃level pavement crack detection on 3D asphalt surfaces using a deep learning network[J].Computer Aided Civil and Infrastructure Engineering,2017,32(10): 805⁃819.
		\bibitem{5} 沙爱民,童 峥,高 杰. 基于卷积神经网络的路表病害识别与测量[J]. 中国公路学报,2018,31(1):1-10. SHA Ai-min, TONG Zheng, GAO Jie. Recognition and Measurement of Pavement Disasters Based on Convolutional Neural Networks [J]. China Journal of Highway and Transport, 2018, 31 (1):1-10.
		\bibitem{6} 徐志刚,赵祥模,宋焕生,雷涛,韦娜.基于直方图估计和形状分析的沥青路面裂缝识别算法[J].仪器仪表学报,2010,31(10):2260-2266.
		\bibitem{7} L. Weiguo, L. Yaru and W. Fang, "Crack detection based on support vector data description," 2017 29th Chinese Control And Decision Conference (CCDC), Chongqing, 2017, pp. 1033-1038, doi: 10.1109/CCDC.2017.7978671.
		\bibitem{8} L. Zhang, F. Yang, Y. Daniel Zhang and Y. J. Zhu, "Road crack detection using deep convolutional neural network," 2016 IEEE International Conference on Image Processing (ICIP), Phoenix, AZ, 2016, pp. 3708-3712, doi: 10.1109/ICIP.2016.7533052.
		\bibitem{9} Yang dan, Zhao Haibin, and Long Zhe, The explanation of MATLAB image processing, 1st edn. Beijing:Tsinghua University Press, 2013.
		\bibitem{10} Gao Haojun, and Du Yuren, “The application of median filtering on image processing”, in Electronic Engineer, Vol.30, pp. 35-36, August 2004
		\bibitem{11} H. Oliveira and P. L. Correia, "Automatic Road Crack Detection and Characterization," in IEEE Transactions on Intelligent Transportation Systems, vol. 14, no. 1, pp. 155-168, March 2013, doi: 10.1109/TITS.2012.2208630.
		\bibitem{12} Tanaka, Naoki \& Uematsu, Kenji. (1998). A Crack Detection Method in Road Surface Images Using Morphology.. 154-157. 
		\bibitem{13} SHORTEN C, KHOSHGOFTAAR T M. A survey on Image Data Augmentation for Deep Learning [J]. Journal of Big Data, 2019, 6(1): 1-48.
		\bibitem{14} Zhang K, Zhang Y, Cheng H, et al. CrackGAN: A Labor-Light Crack Detection Approach Using Industrial Pavement Images Based on Generative Adversarial Learning [J]. arXiv: Computer Vision and Pattern Recognition. arXiv:1909.08216 [cs.CV]. 2019.
		\bibitem{15} LI Liang-fu, SUN Rui-yun. Bridge Crack Detection Algorithm Based on Image Processing under Complex Background [J]. Laser and Optoelectronics Progress. 2019, 56 (6): 112-122.
		\bibitem{16} LI Liang-fu, HU Min. Method for Small-Bridge-Crack Segmentation Based on Generative Adversarial Network [J]. Laser and Optoelectronics Progress. 2019, 56(10): 102-112.
		\bibitem{17} CHAWLA N V, BOWYER K W, HALL L O, et al. SMOTE: Synthetic Minority Over-sampling Technique [J]. Journal of Artificial Intelligence Research, 2002, 16:321-357.
		\bibitem{18} LORE, KIN G, AKINTAYO, et al. LLNet: A Deep Autoencoder Approach to Natural Low-light Image Enhancement [J]. Pattern Recognition, 2015, 61:650-662.
		\bibitem{19} WANG Wan-liang, YANG Xiao-han, ZHAO Yan-wei, et al. Image Enhancement Algorithm with Covolutional Auto-encoder Network [J]. Journal of Zhejiang University (Engineering Science), 2019, 53 (9): 1728-1740.
		\bibitem{20} Young-Shin Lee, Myung-Jee Chung, A study on crack detection using eigenfrequency test data, Computers \& Structures, Volume 77, Issue 3, 2000, Pages 327-342, ISSN 0045-7949, https://doi.org/10.1016/S0045-7949(99)00194-7.
	\end{thebibliography}
\end{document}